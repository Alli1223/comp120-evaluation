\documentclass{scrartcl}

\usepackage[hidelinks]{hyperref}
\usepackage[none]{hyphenat}

\title{Semester One Reflective Report}
\subtitle{COMP120 - Evaluation}

\author{1507516}

\begin{document}

\maketitle

\section{Introduction}
This is my first semester reflective report that will review the key skills that I consider my biggest weaknesses. These are: Time Management, Communication and Reviewing peers code.

\section{Time Management}

%\subsection{Description} 
As most of the work that is set has to be done in our own time, I sometimes find it hard motivate myself to allocate time for studying. Then when the due date nears, I find myself realising there is a lot of work left to do and in some cases, only just finish the assignment on the day it's due.

%\subsection{Interpretation} 
The reason for this is probably related to me spending more time on the tasks that I enjoy, rather than spending more time on the work that is due sooner. 

It is very important for me to improve my ability of managing time effectively, as later the workload will only increase and things like coding take a long time to learn. However recently I have found that using Trello with the Kivy project has helped me a lot to manage my time more effectively, as it helps break projects up into smaller deadlines and requires me to work on the project long before it is due.

%\subsection{Outcome} 
Having experienced Trello on recent projects, I now think that I will be able to apply that knowledge to future projects, which will help motivate me to spend more time studying. 

\section{Communication}
%\subsection{Description} 
I sometimes find it hard to ask my tutors or peers for help, as I will tend to try and solve the problem myself instead of asking for help, as I don't want to ask a stupid question or feel embarrassed. This affects the quality of my submission, as having help with my problems will improve the quality of my work.

%\subsection{Interpretation} 
For me, the most useful elements about the course so far is that the feedback I get is every helpful, and the tutor meetings I have help a lot towards improving the quality of my work. Being able to communicate between myself and the tutors and peers effectively is essential to improving my work in the future. In the first semester I felt too nervous to ask questions, this meant that if I didn't understand what the lecturers were saying, I would just pretend I knew, then go home and research it. This is bad 

%\subsection{Outcome} 
Having analyzed that this is a key issue I need to work on, in the future I will try and prepare questions about anything I'm having problems with at home, then ask them when I get in class. Also over the course of next semester, I will try to address this issue by frequently asking questions in class about things I am unclear on. 

\section{Reviewing Peers Code}
%\subsection{Description} 
I have found it hard to review other peoples work and produce useful comments in the peer reviews. This is mostly due to the code I'm reviewing being hard to comprehend, and far more advanced than my own.

%\subsection{Interpretation}
It is important for me to understand peers work for later on, where I will be working with others on projects and being able to understand and contributing to the code is imperative. However as my ability to read and produce code improves over the course, hopefully so will my ability to critique others code.

%\subsection{Outcome} 
Having experienced several peer review sessions, I now feel more confident in critiquing others work. Furthermore I have learnt that expanding my knowledge of programming has helped in my ability to review others code.
Also by seeing tutors feedback on my work, I was able to lean from their comments and help improve my own feedback to other students.

\section{Conclusion}
These are a few of the key weaknesses I have tried to address that will help me improve for future assignments.


\end{document}
