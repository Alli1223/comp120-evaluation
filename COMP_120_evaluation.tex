\documentclass{scrartcl}

\usepackage[hidelinks]{hyperref}
\usepackage[none]{hyphenat}

\title{Report}
\subtitle{COMP120 - Evlauation}

\author{1507516}

\begin{document}

\maketitle

\abstract{This is my first semester reflective report that will review my key skills that I consider my biggest weaknesses.}

\section{Time Management}

\subsection{Description} 

As most of the work that is set has to be done in our own time, I sometimes find it hard motivate myself to allocate time for studying. And when the due date nears, I find myself realising there is a lot of work left to do, and in some cases, only just finish the assignment on the day it's due.

\subsection{Interpretation} 

This is probably related to me spending more time on the tasks that I enjoy, rather than spending more time on the work that is due sooner. 

It is very important for me to improve my ability of managing time effectively, as later the workload will only increase. 

However recently I have found that using Trello recently with the kivy project has helped me a lot to manage my time more effectvely, as it helpes break projects up into smaller deadlines and requires me to work on the project long before it is due.

\subsection{Outcome} 

Having experienced Trello on more recent projects, I now think that I will be able to apply that knowedge to future projects, which will help motivate me to spend more time studying. 

\section{Communication with Peers and Tutors}
\subsection{Description} 

I sometimes find it hard to ask my tutors or peers for help, as I will tend to try and solve the problem myself instead of asking for help, as I don't want to ask a stupid question or feel embarrassed. This affects the quality of my submission, as having help with my problems will improve the quality of my work.

\subsection{Interpretation} 

For me, the most useful elements about the course so far is that the feedback I get is every helpful, and the tutor meetings I have help alot towards improving the quality of my work. So being able to communicate between myself and the tutors and peers effectivly is essential to improving my work in the future.

At first I felt too nervous to ask questions, however as I get to know my peers and tutors better, I am feeling slightly more confortable asking questions, and being less conserned with how silly the questions might be.

\subsection{Outcome} 

Having analysed that this is a key issue I need to work on, in the future I will try and prepare questions about anything im having problems with at home and ask them when I get in class.

However overall I believe I have slightly improved my ability to communicate with my peers and tutors over the first semester, and hopefully in the future this communication will continue to improve. 

\section{Reviews of Peers Work}

\subsection{Description} 

I have found it hard to review other peoples work and produce useful comments in the peer reviews. This is mostly due to the code im reviewing being hard to comprehend, and far more advanced than my own.

\subsection{Interpretation}

It is important for me to understand peers work for later on, where I will be working with others on projects and being able to understand and contributing to the code is imperative. 

However as my ability to read and produce code improves over the course, hopefully so will my ability to critique others code.

\subsection{Outcome} 

Having experianced several peer review sessions, I now feel more confident in critiquing others work. Furthermore I have learnt that expanding my knowledge of programming has helped in my ability to review others code.

\end{document}
